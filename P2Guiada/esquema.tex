\documentclass[11pt,a4paper]{article}

\usepackage[utf8]{inputenc} 
%\usepackage[T1]{fontenc} 
\usepackage[spanish]{babel}
\decimalpoint
\setlength{\parskip}{0.5\baselineskip} 
\usepackage{fullpage}
\usepackage[procnames]{listings}
\usepackage{fancyhdr}
\usepackage{lastpage}
\usepackage{xcolor}
\usepackage{booktabs}
\usepackage{graphicx}
\usepackage{subcaption}
\usepackage[fleqn]{amsmath}
\parindent 0in 
\setlength{\mathindent}{0pt}
\usepackage{float}

%% DEFINICIONES
\newcommand{\TODO}[1]{{\huge \color{red} \textbf{TODO: }#1 }}
\newcommand{\todo}[1]{{\large \color{red} \textbf{TODO: }#1 }}

\lstset{%
  % backgroundcolor=\color{yellow!20},%
  basicstyle=\ttfamily,%
  numbers=left, numberstyle=\scriptsize, stepnumber=1, numbersep=5pt,%
  frame=single%
}%



\title{Sistemas de Gestión de Datos y de la Información\\Sesión 2 Guiada:
  MongoDB}
\author{Luis M. Costero Valero --- Jesús J. Doménech Arellano}
\date{29 de enero de 2016}

\begin{document} 
\maketitle

\hrule
\textbf{Declaración de autoría:}\\Este documento y el trabajo realizado
 reflejado en él es fruto ÚNICAMENTE del trabajo de sus miembros. Declaramos
 no haber colaborado de ninguna manera con otros grupos, haber compartido el
 código con otros ni haberlo obtenido de una fuente externa.\\
\hrule
\vspace{2cm}
\textbf{Pregunta 1:} El plan ganador es \emph{COLLSCAN} $\rightarrow$
\emph{SORT} (username:1)
\begin{lstlisting}
  "winningPlan" : {
    "stage" : "SORT",
    "sortPattern" : {
      "username" : 1
    },
    "inputStage" : {
      "stage" : "COLLSCAN",
      "filter" : {
	"year" : {
	  "$eq" : 1980
	}
      },
      "direction" : "forward"
    }
  }
\end{lstlisting}
%$
\textbf{Pregunta 2:} El plan ganador es: \emph{IXSCAN} (year\_1)
$\rightarrow$ \emph{FETCH} $\rightarrow$ \emph{KEEP\_MUTATIONS}
$\rightarrow$ \emph{SORT} (username:1).
\begin{lstlisting}
  "winningPlan" : {
    "stage" : "SORT",
    "sortPattern" : {
      "username" : 1
    },
    "inputStage" : {
      "stage" : "KEEP_MUTATIONS",
      "inputStage" : {
	"stage" : "FETCH",
	"inputStage" : {
	  "stage" : "IXSCAN",
	  "keyPattern" : {
	    "year" : 1
	  },
	  "indexName" : "year_1",
	  "isMultiKey" : false,
	  "direction" : "forward",
	  "indexBounds" : {
	    "year" : [
	      "[1980.0, 1980.0]"
	    ]
	  }
	}
      }
    }
  }
\end{lstlisting}

\textbf{Pregunta 3:} El plan ganador es: \emph{IXSCAN} (year\_1) $\rightarrow$
\emph{FETCH}. El número de documentos examinados es 20.000 y se examinan
20.000 claves, se recorre la clave desde MinKey a MaxKey

\begin{lstlisting}
"winningPlan" : {
    "stage" : "FETCH",
    "filter" : {
      "like" : {
	"$eq" : "deportes"
      }
    },
    "inputStage" : {
      "stage" : "IXSCAN",
      "keyPattern" : {
	"year" : 1
      },
      "indexName" : "year_1",
      "isMultiKey" : false,
      "direction" : "forward",
      "indexBounds" : {
	"year" : [
	  "[MinKey, MaxKey]"
	]
      }}
  },
  "executionStats" : {
    "totalKeysExamined" : 20000,
    "totalDocsExamined" : 20000,
    ...
  },
\end{lstlisting}
%$
\textbf{Pregunta 4:} 
\begin{itemize}
 \item El plan ganador es: \emph{IXSCAN} (year\_1\_\_id\_1) $\rightarrow$
   \emph{PROJECTION} (\_id:1)
   \begin{lstlisting}
  "winningPlan" : {
    "stage" : "PROJECTION",
    "transformBy" : {
      "_id" : 1
    },
    "inputStage" : {
      "stage" : "IXSCAN",
      "keyPattern" : {
	"year" : 1,
	"_id" : 1
      },
      "indexName" : "year_1__id_1",
      "isMultiKey" : false,
      "direction" : "forward",
      "indexBounds" : {
	"year" : [
	  "[1980.0, 1980.0]"
	],
	"_id" : [
	  "[MinKey, MaxKey]"
	]
      }
    }
  },
\end{lstlisting}
\item Examina 164 claves y 0 documentos
\begin{lstlisting}
  "totalKeysExamined" : 164,
  "totalDocsExamined" : 0,
\end{lstlisting}
\item El plan rechazado es \emph{IXSCAN} (year\_1) $\rightarrow$
  \emph{FETH} $\rightarrow$ \emph{PROJECTION} (\_id:1)
\begin{lstlisting}
  "rejectedPlans" : [
    {
      "stage" : "PROJECTION",
      "transformBy" : {
	"_id" : 1
      },
      "inputStage" : {
	"stage" : "FETCH",
	"inputStage" : {
	  "stage" : "IXSCAN",
	  "keyPattern" : {
	    "year" : 1
	  },
	  "indexName" : "year_1",
	  "isMultiKey" : false,
	  "direction" : "forward",
	  "indexBounds" : {
	    "year" : [
	      "[1980.0, 1980.0]"
	    ]
	  }
	}
      }
    }
  ]
\end{lstlisting}

\end{itemize}

\textbf{Pregunta 5:} El servidor primario es \emph{:27102} ya que el número
de votos de los tres candidatos es 1 (el mismo), y \emph{:27102} fue el
primero en entrar al set ejecutando ``rs.initiate()'' en una shell
conectada a él.

\textbf{Pregunta 6:} Al desconectar el servidor primario, los servidores
secundarios han detectado que está inalcanzable, por lo que se han decidido
transformar un servidor secundario en primario. Los cambios más importantes
que hay son:
\begin{itemize}
\item La información sobre el servidor que se ha desconectado ha cambiado:\\
  -- Estado del servidor: inalcanzable.\\
  -- Health: 0.
\item La información sobre el nuevo nodo primario (:27103) en este caso,
  es:\\
  -- Estado del servidor: PRIMARY.
\end{itemize}

\textbf{Pregunta 7:}
Tras eliminar los dos servidores primarios, únicamente queda un servidor
alcanzable, pero que mantiene el estado de servidor secundario.

\textbf{Pregunta 8:}
La colección sgdi.users se ha almacenado en el primer servidor creado.

\textbf{Pregunta 9:}
La colección se ha dividido en 14 chunks. El shard 0 y 1 almacenan 5 de
ellos cada uno, mientras que el 2 solamente almacena 4.
El rango de cada uno se puede apreciar en el siguiente código:
\begin{lstlisting}[basicstyle=\tiny\ttfamily]
chunks:
    shard0000	5
    shard0001	5
    shard0002	4
    { "username" : { "$minKey" : 1 } } -->> { "username" : "DooLR" } on : shard0001 Timestamp(2, 0) 
    { "username" : "DooLR" } -->> { "username" : "HiDei" } on : shard0002 Timestamp(3, 0) 
    { "username" : "HiDei" } -->> { "username" : "LVILN" } on : shard0001 Timestamp(4, 0) 
    { "username" : "LVILN" } -->> { "username" : "PLEnl" } on : shard0002 Timestamp(5, 0) 
    { "username" : "PLEnl" } -->> { "username" : "SyjKH" } on : shard0001 Timestamp(6, 0) 
    { "username" : "SyjKH" } -->> { "username" : "Wodbl" } on : shard0002 Timestamp(7, 0) 
    { "username" : "Wodbl" } -->> { "username" : "adQse" } on : shard0001 Timestamp(8, 0) 
    { "username" : "adQse" } -->> { "username" : "eQrFQ" } on : shard0002 Timestamp(9, 0) 
    { "username" : "eQrFQ" } -->> { "username" : "iKtup" } on : shard0001 Timestamp(10, 0) 
    { "username" : "iKtup" } -->> { "username" : "lzycA" } on : shard0000 Timestamp(10, 1) 
    { "username" : "lzycA" } -->> { "username" : "pkkEW" } on : shard0000 Timestamp(1, 10) 
    { "username" : "pkkEW" } -->> { "username" : "tZglH" } on : shard0000 Timestamp(1, 11) 
    { "username" : "tZglH" } -->> { "username" : "xMxcJ" } on : shard0000 Timestamp(1, 12) 
    { "username" : "xMxcJ" } -->> { "username" : { "$maxKey" : 1 } } on : shard0000 Timestamp(1, 13) 

\end{lstlisting}

\newpage{}
\textbf{Pregunta 10:}
\begin{itemize}
\item Consulta 1: \texttt{e.find(\{\_id: 67\})}\\
  -- \emph{Nº shards consultados:} Se consultan los 3 shards creados.
  \\-- \emph{Resultados devueltos cada shard:} El shard 0 y 1 no devuelven ningún
  resultado, mientras que el 3 devuelve un documento.
  \\-- \emph{Tipo de búsqueda:} Cada shard realiza una fase IDHACK $\rightarrow$
  SHARDING\_FILTER. El resutlado de todos los shard se combina en una fase SHARD\_MERGE.
  \\-- \emph{Número de documentos devueltos y examinados :} Únicamente se
  devuelve 1 documento. Se ha consultado únicamente 1 clave y 1 documento
  para realizar la búsqueda.

  \begin{lstlisting}
    "executionStats" : {
      "nReturned" : 1,
      "totalKeysExamined" : 1,
      "totalDocsExamined" : 1,
      "executionStages" : {
        "stage" : "SHARD_MERGE",
        [...]
        "shards" : [
        { "shardName" : "shard0000",
          "executionStages" : {
            "stage" : "SHARDING_FILTER",
            "nReturned" : 0,
            [...]
            "inputStage" : {
              "stage" : "IDHACK",
              "nReturned" : 0,
              [...]
            }}},
        { "shardName" : "shard0001",
          "executionStages" : {
            "stage" : "SHARDING_FILTER",
            "nReturned" : 0,
            [...]
            "inputStage" : {
              "stage" : "IDHACK",
              "nReturned" : 0,
              [...]
            }}},
        { "shardName" : "shard0002",
          "executionStages" : {
            "stage" : "SHARDING_FILTER",
            "nReturned" : 1,
            "inputStage" : {
              "stage" : "IDHACK",
              "nReturned" : 1,
              [...]
            }}}
        ]}
    \end{lstlisting}

\item Consulta 2: \texttt{e.find(\{year:1997\})}\\
  -- \emph{Nº shards consultados:} Se consultan los 3 shards creados.
  \\-- \emph{Resultados devueltos cada shard:} El shard 0 devuelve 288
  documentos, el shard 1 devuelve 304 y el shard 2 devuelve 246 documentos.
  \\-- \emph{Tipo de búsqueda:} Todos los shard han realizado el mismo típo
  de búsqueda: Una fase COLLSCAN $\rightarrow$ SHARDING\_FILTER. Todos los
  resultados de los shards se han combinado mediante una fase SHARD\_MERGE.
  \\-- \emph{Número de documentos devueltos y examinados:} Tras realizar el merge, se
  han devuelto 838 documentos. Se han consultado todos los documentos
  (100000) y ninguna clave.

  \begin{lstlisting}
    "executionStats" : {
      "nReturned" : 838,
      "totalKeysExamined" : 0,
      "totalDocsExamined" : 100000,
      "executionStages" : {
        "stage" : "SHARD_MERGE",
        "nReturned" : 838,
        [...]
        "shards" : [
        { "shardName" : "shard0000",
          "executionStages" : {
            "stage" : "SHARDING_FILTER",
            "nReturned" : 288,
            "inputStage" : {
              "stage" : "COLLSCAN",
              [...]
            },
            [...]
          }}},
      { "shardName" : "shard0001",
        "executionStages" : {
          "stage" : "SHARDING_FILTER",
          "nReturned" : 304,
          "stage" : "COLLSCAN",
          [...]
        },
        [...]
      },
      { "shardName" : "shard0002",
        "executionStages" : {
          "stage" : "SHARDING_FILTER",
          "nReturned" : 246,
          "inputStage" : {
            "stage" : "COLLSCAN",
            [...]
          },
          [...]
        }}
      ]},
  \end{lstlisting}

\item Consulta 3: \texttt{e.find(\{username:''Aaron''\}}\\
  -- \emph{Nº shards consultados:} Únicamente se consulta 1 shard.
  \\-- \emph{Resultados devueltos cada shard:} El shard consultado devuelve
  0 resultados.
  \\-- \emph{Tipo de búsqueda:} El tipo de búsqueda realizada ha sido:
  IXCAN (username\_1) $\rightarrow$ SHARDING\_FILTER $\rightarrow$ FETCH
  $\rightarrow$ SINGLE\_SHARD.
  \\-- \emph{Número de documentos devueltos:} Para realizar la búsqueda se
  han consultado 0 documentos y 0 índices, devolviendo 0 resutlados.


  \begin{lstlisting}
    "executionStats" : {
      "nReturned" : 0,
      "totalKeysExamined" : 0,
      "totalDocsExamined" : 0,
      "executionStages" : {
        "stage" : "SINGLE_SHARD",
        "nReturned" : 0,
        "shards" : [
        { "shardName" : "shard0001",
          "executionStages" : {
            "stage" : "FETCH",
            "nReturned" : 0,
            [...]
            "inputStage" : {
              "stage" : "SHARDING_FILTER",
              "nReturned" : 0,
              "inputStage" : {
                "stage" : "IXSCAN",
                "nReturned" : 0,
                "indexName" : "username_1",
                "indexBounds" : {
                  "username" : [
                  "[\"Aaron\", \"Aaron\"]"
                  ]
                },
                "keysExamined" : 0,
                [...]
              }}}}
        ]},
  \end{lstlisting}
\end{itemize}


\end{document}

