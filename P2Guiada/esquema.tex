\documentclass[11pt,a4paper]{article}

\usepackage[latin1]{inputenc} 
%\usepackage[T1]{fontenc} 
\usepackage[spanish]{babel}
\decimalpoint
\setlength{\parskip}{0.5\baselineskip} 
\usepackage{fullpage}
\usepackage[procnames]{listings}
\usepackage{fancyhdr}
\usepackage{lastpage}
\usepackage{xcolor}
\usepackage{booktabs}
\usepackage{graphicx}
\usepackage{subcaption}
\usepackage[fleqn]{amsmath}
\parindent 0in 
\setlength{\mathindent}{0pt}
\usepackage{float}

%% DEFINICIONES
\newcommand{\TODO}[1]{{\huge \color{red} \textbf{TODO: }#1 }}
\newcommand{\todo}[1]{{\large \color{red} \textbf{TODO: }#1 }}

\lstset{%
  % backgroundcolor=\color{yellow!20},%
  basicstyle=\ttfamily,%
  numbers=left, numberstyle=\scriptsize, stepnumber=1, numbersep=5pt,%
  frame=single%
}%



\title{Sistemas de Gesti�n de Datos y de la Informaci�n\\Sesi�n 2 Guiada:
  MongoDB}
\author{Luis M. Costero Valero --- Jes�s J. Dom�nech Arellano}
\date{22 de enero de 2016}

\begin{document} 
\maketitle

\textbf{Pregunta 1:} El plan ganador es \emph{COLLSCAN} $\rightarrow$
\emph{SORT} (username:1)
\begin{lstlisting}
  "winningPlan" : {
    "stage" : "SORT",
    "sortPattern" : {
      "username" : 1
    },
    "inputStage" : {
      "stage" : "COLLSCAN",
      "filter" : {
	"year" : {
	  "$eq" : 1980
	}
      },
      "direction" : "forward"
    }
  }
\end{lstlisting}
%$
\textbf{Pregunta 2:} El plan ganador es: \emph{IXSCAN} (year\_1)
$\rightarrow$ \emph{FETCH} $\rightarrow$ \emph{KEEP\_MUTATIONS}
$\rightarrow$ \emph{SORT} (username:1).
\begin{lstlisting}
  "winningPlan" : {
    "stage" : "SORT",
    "sortPattern" : {
      "username" : 1
    },
    "inputStage" : {
      "stage" : "KEEP_MUTATIONS",
      "inputStage" : {
	"stage" : "FETCH",
	"inputStage" : {
	  "stage" : "IXSCAN",
	  "keyPattern" : {
	    "year" : 1
	  },
	  "indexName" : "year_1",
	  "isMultiKey" : false,
	  "direction" : "forward",
	  "indexBounds" : {
	    "year" : [
	      "[1980.0, 1980.0]"
	    ]
	  }
	}
      }
    }
  }
\end{lstlisting}

\textbf{Pregunta 3:} El plan ganador es: \emph{IXSCAN} (year\_1) $\rightarrow$
\emph{FETCH}. El n�mero de documentos examinados es 20.000 y se examinan
20.000 claves, se recorre la clave desde MinKey a MaxKey

\begin{lstlisting}
"winningPlan" : {
    "stage" : "FETCH",
    "filter" : {
      "like" : {
	"$eq" : "deportes"
      }
    },
    "inputStage" : {
      "stage" : "IXSCAN",
      "keyPattern" : {
	"year" : 1
      },
      "indexName" : "year_1",
      "isMultiKey" : false,
      "direction" : "forward",
      "indexBounds" : {
	"year" : [
	  "[MinKey, MaxKey]"
	]
      }
    }
  },
  "executionStats" : {
    ...
    "totalKeysExamined" : 20000,
    "totalDocsExamined" : 20000,
    ...
  },
\end{lstlisting}
%$
\textbf{Pregunta 4:} 
\begin{itemize}
 \item El plan ganador es: \emph{IXSCAN} (year\_1\_\_id\_1) $\rightarrow$
   \emph{PROJECTION} (\_id:1)
   \begin{lstlisting}
  "winningPlan" : {
    "stage" : "PROJECTION",
    "transformBy" : {
      "_id" : 1
    },
    "inputStage" : {
      "stage" : "IXSCAN",
      "keyPattern" : {
	"year" : 1,
	"_id" : 1
      },
      "indexName" : "year_1__id_1",
      "isMultiKey" : false,
      "direction" : "forward",
      "indexBounds" : {
	"year" : [
	  "[1980.0, 1980.0]"
	],
	"_id" : [
	  "[MinKey, MaxKey]"
	]
      }
    }
  },
\end{lstlisting}
\item Examina 164 claves y 0 documentos
\begin{lstlisting}
  "totalKeysExamined" : 164,
  "totalDocsExamined" : 0,
\end{lstlisting}
\item El plan rechazado es \emph{IXSCAN} (year\_1) $\rightarrow$
  \emph{FETH} $\rightarrow$ \emph{PROJECTION} (\_id:1)
\begin{lstlisting}
  "rejectedPlans" : [
    {
      "stage" : "PROJECTION",
      "transformBy" : {
	"_id" : 1
      },
      "inputStage" : {
	"stage" : "FETCH",
	"inputStage" : {
	  "stage" : "IXSCAN",
	  "keyPattern" : {
	    "year" : 1
	  },
	  "indexName" : "year_1",
	  "isMultiKey" : false,
	  "direction" : "forward",
	  "indexBounds" : {
	    "year" : [
	      "[1980.0, 1980.0]"
	    ]
	  }
	}
      }
    }
  ]
\end{lstlisting}

\end{itemize}

\textbf{Pregunta 5:} El servidor primario es \emph{:27102} ya que el n�mero
de votos de los tres candidatos es 1 (el mismo), y \emph{:27102} fue el
primero en entrar al set ejecutando ``rs.initiate()'' en una shell
conectada a �l.

\textbf{Pregunta 6:} Al desconectar el servidor primario, los servidores
secundarios han detectado que est� inalcanzable, por lo que se han decidido
transformar un servidor secundario en primario. Los cambios m�s importantes
que hay son:
\begin{itemize}
\item La informaci�n sobre el servidor que se ha desconectado ha cambiado:\\
  -- Estado del servidor: inalcanzable.\\
  -- Health: 0.
\item La informaci�n sobre el nuevo nodo primario (:27103) en este caso,
  es:\\
  -- Estado del servidor: PRIMARY.
\end{itemize}

\textbf{Pregunta 7:}
Tras eliminar los dos servidores primarios, �nicamente queda un servidor
alcanzable, pero que mantiene el estado de servidor secundario.

\textbf{Pregunta 8:}
La colecci�n sgdi.users se ha almacenado en el primer servidor creado.

\textbf{Pregunta 9:}
La colecci�n se ha dividido en 14 chunks. El shard 0 y 1 almacenan 5 de
ellos cada uno, mientras que el 2 solamente almacena 4.
El rango de cada uno se puede apreciar en el siguiente c�digo:
\begin{lstlisting}[basicstyle=\tiny\ttfamily]
chunks:
    shard0000	5
    shard0001	5
    shard0002	4
    { "username" : { "$minKey" : 1 } } -->> { "username" : "DooLR" } on : shard0001 Timestamp(2, 0) 
    { "username" : "DooLR" } -->> { "username" : "HiDei" } on : shard0002 Timestamp(3, 0) 
    { "username" : "HiDei" } -->> { "username" : "LVILN" } on : shard0001 Timestamp(4, 0) 
    { "username" : "LVILN" } -->> { "username" : "PLEnl" } on : shard0002 Timestamp(5, 0) 
    { "username" : "PLEnl" } -->> { "username" : "SyjKH" } on : shard0001 Timestamp(6, 0) 
    { "username" : "SyjKH" } -->> { "username" : "Wodbl" } on : shard0002 Timestamp(7, 0) 
    { "username" : "Wodbl" } -->> { "username" : "adQse" } on : shard0001 Timestamp(8, 0) 
    { "username" : "adQse" } -->> { "username" : "eQrFQ" } on : shard0002 Timestamp(9, 0) 
    { "username" : "eQrFQ" } -->> { "username" : "iKtup" } on : shard0001 Timestamp(10, 0) 
    { "username" : "iKtup" } -->> { "username" : "lzycA" } on : shard0000 Timestamp(10, 1) 
    { "username" : "lzycA" } -->> { "username" : "pkkEW" } on : shard0000 Timestamp(1, 10) 
    { "username" : "pkkEW" } -->> { "username" : "tZglH" } on : shard0000 Timestamp(1, 11) 
    { "username" : "tZglH" } -->> { "username" : "xMxcJ" } on : shard0000 Timestamp(1, 12) 
    { "username" : "xMxcJ" } -->> { "username" : { "$maxKey" : 1 } } on : shard0000 Timestamp(1, 13) 

\end{lstlisting}


\textbf{Pregunta 10:}
\begin{itemize}
\item Consulta 1:
  N� shards consultados:
  \\ Resultados devueltos cada shard:
  \\Tipo de b�squeda:
  \\N�mero de documentos devueltos:
\item Consulta 1:
  N� shards consultados:
  \\Resultados devueltos cada shard:
  \\Tipo de b�squeda:
  \\N�mero de documentos devueltos:
\item Consulta 1:
  N� shards consultados:
  \\Resultados devueltos cada shard:
  \\Tipo de b�squeda:
  \\N�mero de documentos devueltos:
\end{itemize}


\end{document}

