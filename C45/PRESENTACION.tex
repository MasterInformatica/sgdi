\documentclass[10pt]{beamer}

\usetheme{m}

\usepackage{booktabs}
\usepackage[scale=2]{ccicons}

%\usepackage{pgfplots}
%\usepgfplotslibrary{dateplot}

\usepackage{listings}


\title{}
\subtitle{Arboles C4.5 y C5 y otros}
\date{}
\author{Luis María Costero Valero\\Jesús Javier Domenech Arellano}
\institute{Enero 2016}

%\titlegraphic{\hfill\includegraphics[scale=0.4]{bird-illustration-tbb.png}}
\begin{document}

\maketitle


%%======= INDICE ========================================
%\begin{frame}
%  \frametitle{Índice}
%  \setbeamertemplate{section in toc}[sections numbered]
%  \tableofcontents[hideallsubsections]
%\end{frame}

%\section{Introduction} %=== Esrto genera una página de sección.

\begin{frame}
  \frametitle{Intro}
\end{frame}

\begin{frame}
  \frametitle{...}
\end{frame}


\begin{frame}
  \frametitle{Contenidos}
\end{frame}

%===== EJEMPLO =====
\begin{frame}[fragile] % Frame ejemplo 1
  \frametitle{Ejemplo}
\end{frame}

\begin{frame}[fragile] % Frame ejemplo 2
  \frametitle{Ejemplo}
\end{frame}

%======= BIBLIOGRAFÍA =======
\begin{frame}
  \frametitle{Bibliografía}
  
  \begin{enumerate}
  \item \textbf{Wikipedia:}\\\url{https://en.wikipedia.org/}
  \end{enumerate}
\end{frame}

\end{document}
