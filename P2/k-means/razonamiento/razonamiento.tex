\documentclass[11pt,a4paper]{article}

\usepackage[utf8]{inputenc} 
\usepackage[T1]{fontenc} 
\usepackage[spanish]{babel}
\decimalpoint
\setlength{\parskip}{0.5\baselineskip} 
\usepackage{fullpage}
\usepackage[procnames]{listings}
\usepackage{fancyhdr}
\usepackage{lastpage}
\usepackage{xcolor}
\usepackage{graphicx}
\usepackage{subcaption}
\usepackage[fleqn]{amsmath}
\parindent 0in 
\setlength{\mathindent}{0pt}


%% DEFINICIONES
\newcommand{\TODO}[1]{{\huge \color{red} \textbf{TODO: }#1 }}
\newcommand{\todo}[1]{{\large \color{red} \textbf{TODO: }#1 }}



\begin{document} 
\pagestyle{fancy}
\fancyhf{}
\lhead{P2: k-means (aparatado B)}
\rhead{Luis Mª Costero Valero, Jesús J. Doménech Arellano}
\rfoot{\thepage\ / \pageref{LastPage}}
\renewcommand{\headrulewidth}{0.4pt}
\renewcommand{\footrulewidth}{0.4pt}
\section*{}

\textbf{Implementar las 3 medidas de cohesión (radio, diámetro y distancia al
cuadrado promedio con respecto el centroide) para evaluar la calidad del
clustering para valores de k entre 2 y 20. Visualizar la variación de la
cohesión como un gráfico de líneas para cada una de las medidas y razonar
cuál podría ser un buen valor de k.}


En la figuras~\ref{fig:radio},~\ref{fig:diametro}~y~\ref{fig:promedio} se
muestra la evolución de las medidas de radio, diámetro y distancia al
cuadrado promedio según se incrementa el número de clusters (K). La
distancia al cuadrado promedio se ha calculado teniendo en cuenta todas las
instancias con sus respectivos centroides, mientras que el radio y el
diámetro se ha realizado calculando la media aritmética entre el radio (o
diámtero) de cada uno de los clusters. Así mismo, en estas dos últimas
gráficas se muestra la distancia máxima y mínima entre las distancias de
todos los clusters.


\begin{figure}[h]
\begin{subfigure}{0.45\textwidth}
  \centering
 %%----primera subfigura---- 
    \includegraphics[width=0.9\textwidth]{img/radio_maxMin.png}
    \caption{Evolución del radio según aumenta el número de clusters}
    \label{fig:radio} 
\end{subfigure}
\hspace{0.1\textwidth}
\begin{subfigure}{0.45\textwidth}
  %% ----segunda subfigura----
    \includegraphics[width=0.9\textwidth]{img/diametro_maxMin.png}
    \caption{Evolución del diámetro según aumenta el número clusters}
    \label{fig:diametro}
\end{subfigure}
\begin{subfigure}{\textwidth}
  %% ----tercera subfigura----
  \centering
    \includegraphics[width=0.45\textwidth]{img/figure_P.png}
    \caption{Evolución de la distancia al cuadrado promedio según aumenta el número de clusters}
    \label{fig:promedio}
  \end{subfigure}

\end{figure}


Se puede observar como a partir de cierto número de clusters, la distancia
mínima tiende (o es) 0, y la máxima también va decreciendo, ya que existen
clusters donde las instancias coinciden con el centroide o se
encuentran muy próximas a él. \\

Tras realizar el experimento proponemos escoger $K = 8$, dado que
consideramos que las distancias máximas han decrecido lo suficiente en las
tres medidas, y empieza a haber clusters muy pequeños. Además tenemos en
cuenta que a mayor número de clusters objetivo el tiempo de ejecución es
mayor y la mejora no es significativa. Aunque, por supuesto, la elección de
$K$ dependerá en gran medida del problema a resolver.
\end{document}
