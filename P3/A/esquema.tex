\documentclass[11pt,a4paper]{article}

\usepackage[utf8]{inputenc} 
\usepackage[T1]{fontenc} 
\usepackage[spanish]{babel}
\decimalpoint
\setlength{\parskip}{0.5\baselineskip} 
\usepackage{fullpage}
\usepackage[procnames]{listings}
\usepackage{fancyhdr}
\usepackage{lastpage}
\usepackage{xcolor}
\usepackage{graphicx}
\usepackage{subcaption}
n\usepackage[fleqn]{amsmath}
\parindent 0in 
\setlength{\mathindent}{0pt}


%% DEFINICIONES
\newcommand{\TODO}[1]{{\huge \color{red} \textbf{TODO: }#1 }}
\newcommand{\todo}[1]{{\large \color{red} \textbf{TODO: }#1 }}



\begin{document} 
\pagestyle{fancy}
\fancyhf{}
\lhead{P3: MongoDB}
\rhead{Luis Mª Costero Valero, Jesús J. Doménech Arellano}
\rfoot{\thepage\ / \pageref{LastPage}}
\renewcommand{\headrulewidth}{0.4pt}
\renewcommand{\footrulewidth}{0.4pt}
\section*{}

\textbf{Descripción del esquema implícito escogido:}\\

Para el diseño de la base de datos del sitio de preguntas propuesto en la
práctica, se ha considerado la creación de 4 colecciones para almacenar los
usuarios, las preguntas, las contestaciones y comentarios, y las
puntuaciones respectivamente.

\todo{Esto es un draft, cuando esté la bd, pues acabr de escribirlo
  mejor:}\\


Las consideraciones que se han tomado para realizar este esquema han sido:\\

-- Separar preguntas de comentarios y respuestas ya que los comentarios no
tienen un numero acotado, y ademas al contener multimedia van a ser
grandes. (más cosas??)\\
-- La pregunta referencia a las contestaciones para que así se pueda saber
el número de contestaciones sin necesidad de consulatar la coleccion.\\
-- Se ha decidido separar la información de puntuaciones a otra tabla
porque se prevee que va a haber muchas escrituras y lecturas de
estas. Además, esto permite obtener todas las puntutaciones de un usuario
de manera sencilla.\\
-- Aunque las valoraciones están en una coleccion independiente, los
documentos de respuestas lelvan un contador de votos positivos y negativos
para mostrar esa información al consultar una respuesta.\\
-- Los comentarios se encuentran anidados a las respuestas para no tener
que añadir un campo más referenciando a la contestación que comentan, y
además conseguir que la coleccion almacene un único tipo de documento.\\
-- Mas cosas que he pensado (y mucho) pero ahora no me vienen a la cabeza\\




\textbf{USUARIOS:} Contiene:

\todo{Aunque este ejemplo está puesto aquí a pelo, es mejor extraerlo de la
  base de datos}
\begin{lstlisting}
{ "alias": "ShW",
  "Nombre": "Sherlock",
  "Apellidos": "Holmes",
  "experiencia": ["SQL", "PayPal", "Tor"],
  "fecha_creacion": 20150122212536,
  "direccion" : { "calle": "Baker Street",
                  "numero": "221",
                  "letra": "B",
                  "pais": "Inglaterra",
                  "ciudad": "Londres"
                }
}
\end{lstlisting}

\textbf{Preguntas:}
\end{document}



%
%
%%%
%%% Local Variables:
%%% mode: latex
%%% TeX-master: "./esquema.tex"
%%% End:
